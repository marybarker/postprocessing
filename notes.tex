\documentclass{article}
\usepackage{fullpage, amsmath}

\begin{document}

\section{Hybridization}
Hybridization involves the weakening of Continuity 
constraints at inter-element boundaries to allow for: 
\begin{itemize}
\item Larger solution space to achieve approximations in
\item Decoupled systems (parallelizable)
\item Local computations -- better stiffness matrices
\item High order approximations through flux recovery
\end{itemize}
Some of the major hybridized methods are: 
Primal Hybrid, 
Dual/Mixed Hybrid, and 
Lagrangian Discontinuous Galerkin Hybrid. 

For a breakdown of many hybridized methods and a unified framework 
for analyzing some of these, see (Cockburn 1985), 
(Arnold, Brezzi, Cockburn, Marini 2001/2), (Cockburn et al. 2016). 
In this work, we will focus on the Primal Hybrid formulation, with a 
specific post-processing technique for improved accuracy. 

The model problem for this study is $$-d^*du=f.$$ We will look at 
2 and 3 dimension versions with $d^*d$ as div grad, grad div, and curl curl. 

The primal formulation for this problem is derived as follows: 
Solve $-d^*du=f$ weakly. i.e. $\langle-d^*du,v\rangle=\langle f,v\rangle$ 
for all $v\in V$. This, in turn, can be written as a first-order system 
using the weak form:
$$
\langle du,dv\rangle-\langle du, v\rangle_\partial=\langle f,v\rangle
$$
where the $\langle,\rangle_\partial$ denotes inner product on the 
boundaries. The explicit formulation of this term comes from the 
Divergence/Stokes/Green's theorem and depends
on the dimension of the domain and on the derivative $d$. 

\section{Postprocessing}
With primal hybrid method, we end up with an approximate 
solution $u_h\in V_h$ to the function $-d^*du=f$. 
In most physically relevant problems, however, we are looking both for $u$ and for 
$\sigma=-du$. 

Now, rather than evaluating a numerical derivative of the 
computed solution $u_h$ and leaving it at that, we post-process to achieve 
a much higher order of accuracy solution $\sigma_h$ for $-du$. 

There are many methods used for improving the approximation for $\sigma_h$ 
using the information gained from solving for $u_h$. 
See, e.g., (Chou, Kwak, and Kim, 2002). 
The one that we are 
considering here uses the trace values as well as computed values for $u_h$. 

So, given the approximation $u_h$, we then solve the minimization problem:
minimize the functional 
$$J(\sigma)=\frac{1}{2}\left(||\sigma+du_h||^2+||d\sigma-f||^2\right)$$
over all $\sigma\in\Sigma_h$ 
subject to the constraints: 
$$\begin{array}{rcl}
	\langle du_h,dv\rangle_K+\langle v,\sigma\rangle_{\partial K}
		&=&\langle f,v\rangle_K\\
	\langle u,\tau\rangle_{\partial K}&=&0
\end{array}~~\forall (v,\sigma)\in W_h\times\Sigma_h$$
Here the equality is taken over each element $K\in T_h$, and 
$W_h$ the approximation function space for $u$ has weakened continuity 
constraints on the inter-element boundaries. This allows for a better 
approximation for $\sigma$ while the boundary terms introduce lagrange 
multipliers that insure that $u_h$ satisfies the equations.  

\section{Lagrange Multipliers}
Recall the basic idea of Lagrange Multipliers from Calculus: 
to minimize $F(x)$ subject to the constraint $G(x) = H(x)$, 
solve $\nabla F=\lambda\nabla G$ 

In this case, we want to solve $\nabla J(\sigma)=0$ while enforcing 
the original equation constraints on $\sigma$. 
So we solve for $(\sigma_h, u_h)$: 
$$\langle\nabla J(\sigma_h), \tau\rangle = 0~~\forall\tau\in\Sigma_h$$
and at the same time:  
$$
\langle u_h,v\rangle-\langle v,\sigma_h\rangle_\partial=\langle f,v\rangle,~~~
\langle u_h,\tau\rangle_\partial = 0
$$


\end{document}
